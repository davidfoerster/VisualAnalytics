\documentclass{beamer}

\usepackage{amsmath}
\usepackage{amssymb}
%\usepackage{textcomp}
\usepackage[ngerman]{babel}
\usepackage{caption}
\usepackage{graphicx}
\usepackage{color}
\usepackage{listings}
\usepackage{lmodern}
\usepackage[T1]{fontenc}
\usepackage[utf8x]{inputenc}

\graphicspath{{./imgs/}}

% customized theme
\usetheme{CambridgeUS}
\usecolortheme{whale}
\beamertemplatenavigationsymbolsempty
\setbeamertemplate{footline}[page number]
\setbeamercolor{frametitle}{fg=white,bg=structure}
\setbeamercolor{theorem}{fg=black,bg=structure!30}
\setbeamercovered{transparent}

\definecolor{mygreen}{rgb}{0, 0.6, 0}
\definecolor{mymauve}{rgb}{0.58, 0, 0.82}

\lstset{
	language=Python,
	breaklines=true,
	tabsize=4,
	%basicstyle=\ttfamily,
	keywordstyle=\color{blue},
	stringstyle=\color{mymauve},
	commentstyle=\color{mygreen},
	otherkeywords={
		None,False,True,
		self,
		bytes}
}

% meta informations for title page
\title{Abschlusspräsentation}
\subtitle{Visual Analytics für raumzeitliche Daten}
\author{Christian Diehr \and Benjamin Drost \and David Foerster}
\institute{Institut für Informatik\\Humboldt-Universität zu Berlin}
\logo{\includegraphics[width=2cm,height=2cm]{imgs/hulogo.pdf}}
\date{16. Februar 2016}

\begin{document}

    \begin{frame}
        \titlepage
    \end{frame}
    \logo % logo only appears on title page
        
    \section{Clustering}
	% Erklären, was Clustering ist
	% Erklären, wie der kMeans-Algorithmus funktioniert
	% KMeans einfach, häufig verwendet
	% Abhängigkeit von gewähltem Startpunkt als Nachteil
    \begin{frame}
			\frametitle{k-Means-Algorithmus}
			\begin{figure}
				\centering\includegraphics[width=.7\textwidth]{kmeans.png}
				\setbeamertemplate{caption}{\raggedright\insertcaption\par}
				\caption{Quelle: pypr.sourceforge.net/\_images}
				\setbeamertemplate{caption}[default]
			\end{figure}
    \end{frame}
    
%    \section{Histogramm}
%    \begin{frame}
%			\frametitle{Histogramm-Optionen}
%			\begin{itemize}
%				\setlength\itemsep{1em}
%				\item Im Histogramm-Dialog Einstellungen auswählen
%				\begin{itemize}
%					\setlength\itemsep{1em}
%					\item Intervallgröße festlegen 
%					\item Monat wählen
%					\item Darzustellende Partikel bestimmen				
%				\end{itemize}
%				\item Darstellung per matplotlib
%				\item Tooltip erscheint bei Berührung einer Bar im Histogramm				
%			\end{itemize}
%    \end{frame}

\end{document}
