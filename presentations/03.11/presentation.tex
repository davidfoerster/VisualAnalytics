\documentclass{beamer}

\usepackage{amsmath}
\usepackage{amssymb}
\usepackage[ngerman]{babel}
\usepackage{graphicx}
\usepackage[T1]{fontenc}
\usepackage[utf8x]{inputenc}

\graphicspath{{./imgs/}}

% customized theme
\usetheme{CambridgeUS}
\usecolortheme{whale}
\beamertemplatenavigationsymbolsempty
\setbeamertemplate{footline}[page number]
\setbeamercolor{frametitle}{fg=white,bg=structure}
\setbeamercolor{theorem}{fg=black,bg=structure!30}
\setbeamercovered{transparent}

% meta informations for title page
\title{Übung 2}
\subtitle{Visual Analytics für raumzeitliche Daten}
\author{Christian Diehr \and Benjamin Drost \and David Foerster}
\institute{Institut für Informatik\\Humboldt-Universität zu Berlin}
\logo{\pgfimage[width=2cm,height=2cm]{./imgs/hulogo}}
\date{03. November 2015}

\begin{document}

    \begin{frame}
        \titlepage
    \end{frame}
    \logo{} % logo only appears on title page
    
    \section{Zwischenpräsentation}
    \begin{frame}{Werkzeuge}
        \begin{itemize}
	        \setlength\itemsep{1em}
        	\item Python (gute Anbindung an bestehende Pakete)
        	\item PyQt (aktive Pflege)
        	\item PyQtGraph (intuitiv für Python/Qt-Entwickler)
        	\item NumPy (wesentlich für wissenschaftliche Berechnungen)
        	\item PyCharm (IDE)
        \end{itemize}
    \end{frame}
    
    \begin{frame}{Implementierung}
    	\begin{itemize}
	        \setlength\itemsep{1em}
	        \item Einarbeitung in Python und Qt
	        \item Anbindung der Bibliotheken und Frameworks untereinander
    		\item Kommunikation der grafischen Bedienelemente
    	\end{itemize}
    \end{frame}
    
    \begin{frame}{Gruppenarbeit}
    	\begin{itemize}
    		\setlength\itemsep{1em}
    		\item David: XY-Plot, Auswahl von Punkten, Tooltips
    		\item Christian: Filter, Statistiken, Historie
    		\item Ben: Timeline, Zoom
    		\item[]
    		\item Aufgaben 1 -- 5 bis ca. 11. Nov.
    		\item Bonus soweit man kommt
    	\end{itemize}
    \end{frame}
    
\end{document}
