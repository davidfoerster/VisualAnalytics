\documentclass{beamer}

\usepackage{amsmath}
\usepackage{amssymb}
\usepackage[ngerman]{babel}
\usepackage{graphicx}
\usepackage[T1]{fontenc}
\usepackage[utf8x]{inputenc}

\graphicspath{{./imgs/}}

% customized theme
\usetheme{CambridgeUS}
\usecolortheme{whale}
\beamertemplatenavigationsymbolsempty
\setbeamertemplate{footline}[page number]
\setbeamercolor{frametitle}{fg=white,bg=structure}
\setbeamercolor{theorem}{fg=black,bg=structure!30}
\setbeamercovered{transparent}

% meta informations for title page
\title{Übung 2}
\subtitle{Visual Analytics für raumzeitliche Daten}
\author{Christian Diehr \and Benjamin Drost \and David Foerster}
\institute{Institut für Informatik\\Humboldt-Universität zu Berlin}
\logo{\pgfimage[width=2cm,height=2cm]{./imgs/hulogo}}
\date{03. November 2015}

\begin{document}

    \begin{frame}
        \titlepage
    \end{frame}
    \logo{} % logo only appears on title page
    
    \section{Zwischenpräsentation}
    \begin{frame}{Werkzeuge}
        \begin{itemize}
	        \setlength\itemsep{1em}
        	\item TODO: benutzte Entwicklungswerkzeuge
        	\item TODO: Vorteile dieser Werkzeuge
        	\item TODO: externe Bibliotheken
        \end{itemize}
    \end{frame}
    
    \begin{frame}{Implementierung}
    	\begin{itemize}
	        \setlength\itemsep{1em}
    		\item TODO: welche Algs. und Vis. eigenständig implementieren?
    		\item TODO: Aufwand in Stunden
    	\end{itemize}
    \end{frame}
    
    \begin{frame}{Gruppenarbeit}
    	\begin{itemize}
    		\setlength\itemsep{1em}
    		\item TODO: Aufteilung der Arbeit
    		\item TODO: Zeitplanung
    	\end{itemize}
    \end{frame}
    
\end{document}
