\documentclass{beamer}

\usepackage{amsmath}
\usepackage{amssymb}
%\usepackage{textcomp}
\usepackage[ngerman]{babel}
\usepackage{graphicx}
\usepackage{color}
\usepackage{listings}
\usepackage[absolute,overlay]{textpos}
\usepackage{lmodern}
\usepackage[T1]{fontenc}
\usepackage[utf8x]{inputenc}

\graphicspath{{./imgs/}}

% customized theme
\usetheme{CambridgeUS}
\usecolortheme{whale}
\beamertemplatenavigationsymbolsempty
\setbeamertemplate{footline}[page number]
\setbeamercolor{frametitle}{fg=white,bg=structure}
\setbeamercolor{theorem}{fg=black,bg=structure!30}
\setbeamercovered{transparent}

\definecolor{mygreen}{rgb}{0, 0.6, 0}
\definecolor{mymauve}{rgb}{0.58, 0, 0.82}

\lstset{
	language=Python,
	breaklines=true,
	tabsize=4,
	%basicstyle=\ttfamily,
	keywordstyle=\color{blue},
	stringstyle=\color{mymauve},
	commentstyle=\color{mygreen},
	otherkeywords={
		None,False,True,
		self,
		bytes
	}
}

\newcommand{\wholeslidegraphics}[1]{
	\begin{center}
		\includegraphics[width=0.90\textwidth,height=0.95\textheight,keepaspectratio]{#1}
	\end{center}
}
\newcommand{\graphicssource}[1]{
	\begin{textblock*}{4cm}(8.7cm,8.6cm)
		\begin{beamercolorbox}[ht=0.5cm,right]{framesource}
				\usebeamerfont{framesource}\usebeamercolor[fg]{framesource} Quelle: {#1}
		\end{beamercolorbox}
	\end{textblock*}
}

\setbeamercolor{framesource}{fg=gray}
\setbeamerfont{framesource}{size=\tiny}

% meta informations for title page
\title{Übung 5, Nachtrag}
\subtitle{Visual Analytics für raumzeitliche Daten}
\author{Christian Diehr \and Benjamin Drost \and David Foerster}
\institute{Institut für Informatik\\Humboldt-Universität zu Berlin}
\logo{\includegraphics[width=2cm,height=2cm]{../hulogo}}
\date{19. Januar 2016}

\begin{document}

\begin{frame}
		\titlepage
\end{frame}
\logo % logo only appears on title page

\section{Erweiterung: Option 2}
\begin{frame}
	\frametitle{Zielstellung}
	\begin{center}
		\Large
		Ähnlichkeitsanalyse von Korngrößenverteilungen verschiedener Signalepisoden
	\end{center}
\end{frame}

\begin{frame}
	\frametitle{Anforderungen/Bewertung}
	\begin{description}
		\item[(o)] muss
		\item[(+)] soll
		\item[(++)] kann
		\item[(-)] soll nicht
	\end{description}
\end{frame}

\begin{frame}
	%\frametitle{Ähnlichkeitsanalyse von Korngrößenverteilungen verschiedener Signalepisoden}
	\begin{itemize}
		\setlength\itemsep{1em}
		\item Darstellung als vielfache kleine Histogramme der Episoden
		\item wählbares (+) oder adaptives (++) Binning der Korngrößen für verschiedene Histogrammgrößen
    \item hierarchisches Clustering nach einem festen (+) oder einem wählbaren (++) Abstandsmaß
    \item total geordnete, „lineare“ (o) oder hierarchische Anordnung (++) der Histogramme
    \item Abstände zwischen Verteilungen werden nicht grafisch (z. B. durch Abstände zwischen Histogrammen) kodiert (-) sondern bestenfalls numerisch dargestellt
    \item eine einzige Zeitleiste hebt farblich kodiert die relevanten Zeitausschnitte für jedes Histogramm hervor (+)
	\end{itemize}
\end{frame}
\begin{frame}
	\wholeslidegraphics{histogram-small-multiples}
	\graphicssource{[Mey]}
\end{frame}

\section{Bildquellen}
\begin{frame}
	\frametitle{Bildquellen}
	\begin{itemize}
		\item{[Mey] \href{http://viz.sdql.com/2014/10/31/small-multiples-starters-innings-pitched-for-each-mlb-teams-in-2013/}{Sports Data Visualization, 2013} (Meyer, Joe)}
	\end{itemize}
\end{frame}

\end{document}
