\documentclass{beamer}

\usepackage{amsmath}
\usepackage{amssymb}
%\usepackage{textcomp}
\usepackage[ngerman]{babel}
\usepackage{graphicx}
\usepackage{color}
\usepackage{listings}
\usepackage{lmodern}
\usepackage[T1]{fontenc}
\usepackage[utf8x]{inputenc}

\graphicspath{{./imgs/}}

% customized theme
\usetheme{CambridgeUS}
\usecolortheme{whale}
\beamertemplatenavigationsymbolsempty
\setbeamertemplate{footline}[page number]
\setbeamercolor{frametitle}{fg=white,bg=structure}
\setbeamercolor{theorem}{fg=black,bg=structure!30}
\setbeamercovered{transparent}

\definecolor{mygreen}{rgb}{0, 0.6, 0}
\definecolor{mymauve}{rgb}{0.58, 0, 0.82}

\lstset{
	language=Python,
	breaklines=true,
	tabsize=4,
	%basicstyle=\ttfamily,
	keywordstyle=\color{blue},
	stringstyle=\color{mymauve},
	commentstyle=\color{mygreen},
	otherkeywords={
		None,False,True,
		self,
		bytes}
}

\newcommand{\wholeslidegraphics}[1]{
	\begin{center}
		\includegraphics[width=0.90\textwidth,height=0.95\textheight,keepaspectratio]{#1}
	\end{center}
}

% meta informations for title page
\title{Übung 5}
\subtitle{Visual Analytics für raumzeitliche Daten}
\author{Christian Diehr \and Benjamin Drost \and David Foerster}
\institute{Institut für Informatik\\Humboldt-Universität zu Berlin}
\logo{\includegraphics[width=2cm,height=2cm]{../hulogo}}
\date{12. Januar 2016}

\begin{document}

    \begin{frame}
        \titlepage
    \end{frame}
    \logo % logo only appears on title page

    \section{Skalierbarkeit}
    \begin{frame}
			\frametitle{Skalierbarkeit: Kreise}
			\begin{itemize}
				\setlength\itemsep{1em}
				\item Aggregation von Datenpunkten in \emph{runden} Flächen
				\item Punktanzahl: Radius
				\item Mittelwert: Zentrum
				\item Varianz: Farbskala (algorithmisch angepasst)
					\begin{itemize}
						\item erfordert neue graphische Variable(n) für Klassenzugehörigkeit
					\end{itemize}
				\item geschätzter Arbeitsaufwand: 10\,h
			\end{itemize}
    \end{frame}
    \begin{frame}
    	\wholeslidegraphics{scatter-plot-weighted}
    \end{frame}

    \begin{frame}
			\frametitle{Skalierbarkeit: Parkett}
			\begin{itemize}
				\setlength\itemsep{1em}
				\item Aggregation von Datenpunkten in \emph{platonischem Parkett}
				\item Punktanzahl: Farbintensität/Helligkeit
				\item Mittelwert: Zentrum
				\item Varianz: Farbwert
				\item geschätzter Arbeitsaufwand: 10\,h
			\end{itemize}
    \end{frame}
    \begin{frame}
    	\wholeslidegraphics{scatter-plot-compressed}
    \end{frame}
    \begin{frame}
    	\wholeslidegraphics{scatter-heat-plot}
    \end{frame}

    \section{Semantischer Zoom}
    \begin{frame}
			\frametitle{Semantischer Zoom}
			\begin{itemize}
				\setlength\itemsep{1em}
				\item Ideen zur Kodierung zusätzliche Dimensionen bei angemessener Vergrößerungsstufe, z.\,B.
					\begin{itemize}
						\item farbliche Zeitskala ordnet Punkte im Jahresverlauf ein
						\item Luftfeuchtigkeit (Farbe oder Form)
						\item vergangene Zeit seit Anfang/Ende/Mitte eines \emph{wet deposit event} (Farbe oder Form)
					\end{itemize}
				\item zusätzlich numerische Angabe (bei Parkettdarstellung)
				\item gesch.\ Aufwand: je 1--2\,h (ohne Datenaufbereitung)
			\end{itemize}
    \end{frame}
    \begin{frame}
    	\wholeslidegraphics{scatter-plot-semantic-zoom}
    \end{frame}

    \section{Erweiterung: Option 2}
    \begin{frame}
			\frametitle{Erweiterung: Option 2}
			\begin{itemize}
				\setlength\itemsep{1em}
				\item \emph{binning} je nach Größenklassenanzahl
				\item Darstellung der Histogramme für verschiedener Zeitabschnitte
					\begin{itemize}
						\item in einem einzelnen Diagramm
						\item als kleine Vielfache
							\begin{itemize}
								\item viel Spielraum für Größen- und Genauigkeitsstufen
								\item achronologische Anordnung (Folge, Hierarchie) möglich, z.\,B. nach Momenten oder Ähnlichkeit (zu einer kanonischen Verteilung oder untereinander)
							\end{itemize}
						\item interaktiv im Zeitverlauf mit gleitendem Fenster
					\end{itemize}
				\item gesch.\ Aufwand: 4--6\,h je Darstellungstyp (ohne Datenaufbereitung)
			\end{itemize}
    \end{frame}
    \begin{frame}
    	\wholeslidegraphics{overlay-histogram}
    \end{frame}
    \begin{frame}
    	\wholeslidegraphics{histogram-small-multiples}
    \end{frame}

    \section{Bildquellen}
    \begin{frame}
    	\frametitle{Bildquellen}
    	\begin{itemize}
    		\item folgen \ldots
    	\end{itemize}
    \end{frame}

\end{document}
