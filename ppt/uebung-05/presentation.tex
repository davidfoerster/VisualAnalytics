\documentclass{beamer}

\usepackage{amsmath}
\usepackage{amssymb}
%\usepackage{textcomp}
\usepackage[ngerman]{babel}
\usepackage{graphicx}
\usepackage{color}
\usepackage{listings}
\usepackage[absolute,overlay]{textpos}
\usepackage{lmodern}
\usepackage[T1]{fontenc}
\usepackage[utf8x]{inputenc}

\graphicspath{{./imgs/}}

% customized theme
\usetheme{CambridgeUS}
\usecolortheme{whale}
\beamertemplatenavigationsymbolsempty
\setbeamertemplate{footline}[page number]
\setbeamercolor{frametitle}{fg=white,bg=structure}
\setbeamercolor{theorem}{fg=black,bg=structure!30}
\setbeamercovered{transparent}

\definecolor{mygreen}{rgb}{0, 0.6, 0}
\definecolor{mymauve}{rgb}{0.58, 0, 0.82}

\lstset{
	language=Python,
	breaklines=true,
	tabsize=4,
	%basicstyle=\ttfamily,
	keywordstyle=\color{blue},
	stringstyle=\color{mymauve},
	commentstyle=\color{mygreen},
	otherkeywords={
		None,False,True,
		self,
		bytes
	}
}

\newcommand{\wholeslidegraphics}[1]{
	\begin{center}
		\includegraphics[width=0.90\textwidth,height=0.95\textheight,keepaspectratio]{#1}
	\end{center}
}
\newcommand{\graphicssource}[1]{
	\begin{textblock*}{4cm}(8.7cm,8.6cm)
		\begin{beamercolorbox}[ht=0.5cm,right]{framesource}
				\usebeamerfont{framesource}\usebeamercolor[fg]{framesource} Quelle: {#1}
		\end{beamercolorbox}
	\end{textblock*}
}

\setbeamercolor{framesource}{fg=gray}
\setbeamerfont{framesource}{size=\tiny}

% meta informations for title page
\title{Übung 5}
\subtitle{Visual Analytics für raumzeitliche Daten}
\author{Christian Diehr \and Benjamin Drost \and David Foerster}
\institute{Institut für Informatik\\Humboldt-Universität zu Berlin}
\logo{\includegraphics[width=2cm,height=2cm]{../hulogo}}
\date{12. Januar 2016}

\begin{document}

\begin{frame}
		\titlepage
\end{frame}
\logo % logo only appears on title page

\section{Skalierbarkeit}
\begin{frame}
	\frametitle{Skalierbarkeit: Kreise}
	\begin{itemize}
		\setlength\itemsep{1em}
		\item Aggregation von Datenpunkten in \emph{runden} Flächen
		\item Punktanzahl: Radius
		\item Mittelwert: Zentrum
		\item Varianz: Farbskala (algorithmisch angepasst)
			\begin{itemize}
				\item erfordert neue graphische Variable(n) für Klassenzugehörigkeit
			\end{itemize}
		\item geschätzter Arbeitsaufwand: 10\,h
	\end{itemize}
\end{frame}
\begin{frame}
	\wholeslidegraphics{scatter-plot-weighted}
	\graphicssource{[STATA]}
\end{frame}

\begin{frame}
	\frametitle{Skalierbarkeit: Parkett}
	\begin{itemize}
		\setlength\itemsep{1em}
		\item Aggregation von Datenpunkten in \emph{platonischem Parkett}
		\item Punktanzahl: Farbintensität/Helligkeit
		\item Mittelwert: Zentrum
		\item Varianz: Farbwert
		\item geschätzter Arbeitsaufwand: 10\,h
	\end{itemize}
\end{frame}
\begin{frame}
	\wholeslidegraphics{scatter-plot-compressed}
	\graphicssource{[Qlik]}
\end{frame}
\begin{frame}
	\wholeslidegraphics{scatter-heat-plot}
	\graphicssource{[Gib]}
\end{frame}

\section{Semantischer Zoom}
\begin{frame}
	\frametitle{Semantischer Zoom}
	\begin{itemize}
		\setlength\itemsep{1em}
		\item Ideen zur Kodierung zusätzliche Dimensionen bei angemessener Vergrößerungsstufe, z.\,B.
			\begin{itemize}
				\item farbliche Zeitskala ordnet Punkte im Jahresverlauf ein
				\item Luftfeuchtigkeit (Farbe oder Form)
				\item vergangene Zeit seit Anfang/Ende/Mitte eines \emph{wet deposit event} (Farbe oder Form)
			\end{itemize}
		\item zusätzlich numerische Angabe (bei Parkettdarstellung)
		\item gesch.\ Aufwand: je 1--2\,h (ohne Datenaufbereitung)
	\end{itemize}
\end{frame}
\begin{frame}
	\wholeslidegraphics{scatter-plot-semantic-zoom}
	\graphicssource{[Qlik]}
\end{frame}

\section{Erweiterung: Option 2}
\begin{frame}
	\frametitle{Erweiterung: Option 2}
	\begin{itemize}
		\setlength\itemsep{1em}
		\item \emph{binning} je nach Größenklassenanzahl
		\item Darstellung der Histogramme für verschiedener Zeitabschnitte
			\begin{itemize}
				\item in einem einzelnen Diagramm
				\item als kleine Vielfache
					\begin{itemize}
						\item viel Spielraum für Größen- und Genauigkeitsstufen
						\item achronologische Anordnung (Folge, Hierarchie) möglich, z.\,B. nach Momenten oder Ähnlichkeit (zu einer kanonischen Verteilung oder untereinander)
					\end{itemize}
				\item interaktiv im Zeitverlauf mit gleitendem Fenster
			\end{itemize}
		\item gesch.\ Aufwand: 4--6\,h je Darstellungstyp (ohne Datenaufbereitung)
	\end{itemize}
\end{frame}
\begin{frame}
	\wholeslidegraphics{overlay-histogram}
	\graphicssource{[Mat]}
\end{frame}
\begin{frame}
	\wholeslidegraphics{histogram-small-multiples}
	\graphicssource{[Mey]}
\end{frame}

\section{Bildquellen}
\begin{frame}
	\frametitle{Bildquellen}
	\begin{itemize}
		\item{[STATA] \href{http://www.stata.com/support/faqs/graphics/gph/graphdocs/scatterplot-with-weighted-markers/}{STATA FAQ}}
		\item{[Qlik] \href{https://help.qlik.com/sense/2.1/en-US/online/Subsystems/Hub/Content/Visualizations/ScatterPlot/scatter-plot.htm}{Qlik Sense Help}}
		\item{[Gib] \href{http://www.gibmetportal.gi/data/openair?build=scatterplot}{Gibraltar Meteorological Data Portal}}
		\item{[Mat] \href{http://blogs.sas.com/content/graphicallyspeaking/2013/11/21/comparative-histograms/}{Graphically Speaking, 2013} (Matange, Satay)}
		\item{[Mey] \href{http://viz.sdql.com/2014/10/31/small-multiples-starters-innings-pitched-for-each-mlb-teams-in-2013/}{Sports Data Visualization, 2013} (Meyer, Joe)}
	\end{itemize}
\end{frame}

\end{document}
