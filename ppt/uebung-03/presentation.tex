\documentclass{beamer}

\usepackage{amsmath}
\usepackage{amssymb}
%\usepackage{textcomp}
\usepackage[ngerman]{babel}
\usepackage{graphicx}
\usepackage{color}
\usepackage{listings}
\usepackage{lmodern}
\usepackage[T1]{fontenc}
\usepackage[utf8x]{inputenc}

\graphicspath{{./imgs/}}

% customized theme
\usetheme{CambridgeUS}
\usecolortheme{whale}
\beamertemplatenavigationsymbolsempty
\setbeamertemplate{footline}[page number]
\setbeamercolor{frametitle}{fg=white,bg=structure}
\setbeamercolor{theorem}{fg=black,bg=structure!30}
\setbeamercovered{transparent}

\definecolor{mygreen}{rgb}{0, 0.6, 0}
\definecolor{mymauve}{rgb}{0.58, 0, 0.82}

\lstset{
	language=Python,
	breaklines=true,
	tabsize=4,
	%basicstyle=\ttfamily,
	keywordstyle=\color{blue},
	stringstyle=\color{mymauve},
	commentstyle=\color{mygreen},
	otherkeywords={
		None,False,True,
		self,
		bytes}
}

% meta informations for title page
\title{Übung 2}
\subtitle{Visual Analytics für raumzeitliche Daten}
\author{Christian Diehr \and Benjamin Drost \and David Foerster}
\institute{Institut für Informatik\\Humboldt-Universität zu Berlin}
\logo{\includegraphics[width=2cm,height=2cm]{../hulogo}}
\date{08. Dezember 2015}

\begin{document}

    \begin{frame}
        \titlepage
    \end{frame}
    \logo % logo only appears on title page

    \section{Aufgabe 1}
    \begin{frame}
	    \frametitle{FitLine}
			\begin{itemize}
			\item Berechnung der FitLine mit Hilfe der Least-Square Methode
			\end{itemize}
    \end{frame}

    \section{Aufgabe 2}
    \begin{frame}[containsverbatim]
			\frametitle{Tooltip}
			\begin{itemize}
				\setlength\itemsep{1em}
				\item Tooltip setzen wie in Übung 2 Aufgabe 2
			\end{itemize}
			\begin{lstlisting}
setToolTip('a=%f\nb=%f\np=%f\nr=%f' % (a, b, q, r))
			\end{lstlisting}
    \end{frame}


    \section{Aufgabe 3}
    \begin{frame}
			\frametitle{Regressionsgeraden}
			\begin{itemize}
				\setlength\itemsep{1em}
				\item wie Aufgabe 1, aber auf den derzeit ausgewählten Punkten
				\item aktualisiere Regressionsgraph bei Änderung der Auswahl
			\end{itemize}
    \end{frame}


%    \section{Aufgabe 4}
%    \begin{frame}[containsverbatim]
%    	\frametitle{Explorationsloop}
%    	\begin{itemize}
%    		\item
%    	\end{itemize}
%    \end{frame}

    \section{Aufgabe 8}
    \begin{frame}
			\frametitle{Unsicherheitsbereich der Regression}
			\begin{itemize}
				\item Füge zwei weitere Linien mit Parametern ein, zu denen die Fehlerspanne addiert bzw. subtrahiert wird
			\end{itemize}
    \end{frame}

\end{document}
