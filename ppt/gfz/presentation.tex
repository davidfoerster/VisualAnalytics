\documentclass{beamer}

\usepackage{amsmath}
\usepackage{amssymb}
%\usepackage{textcomp}
\usepackage[ngerman]{babel}
\usepackage{graphicx}
\usepackage{color}
\usepackage{listings}
\usepackage{lmodern}
\usepackage[T1]{fontenc}
\usepackage[utf8x]{inputenc}

\graphicspath{{./imgs/}}

% customized theme
\usetheme{CambridgeUS}
\usecolortheme{whale}
\beamertemplatenavigationsymbolsempty
\setbeamertemplate{footline}[page number]
\setbeamercolor{frametitle}{fg=white,bg=structure}
\setbeamercolor{theorem}{fg=black,bg=structure!30}
\setbeamercovered{transparent}

\definecolor{mygreen}{rgb}{0, 0.6, 0}
\definecolor{mymauve}{rgb}{0.58, 0, 0.82}

\lstset{
	language=Python,
	breaklines=true,
	tabsize=4,
	%basicstyle=\ttfamily,
	keywordstyle=\color{blue},
	stringstyle=\color{mymauve},
	commentstyle=\color{mygreen},
	otherkeywords={
		None,False,True,
		self,
		bytes}
}

% meta informations for title page
\title{Funktionen des Prototyps}
\subtitle{Visual Analytics für raumzeitliche Daten}
\author{Christian Diehr \and Benjamin Drost \and David Foerster}
\institute{Institut für Informatik\\Humboldt-Universität zu Berlin}
\logo{\includegraphics[width=2cm,height=2cm]{../hulogo}}
\date{22. Dezember 2015}

\begin{document}

    \begin{frame}
        \titlepage
    \end{frame}
    \logo % logo only appears on title page
    
    \section{}
    \begin{frame}
	    \frametitle{Gliederung}
			\begin{itemize}
			\item Main-Window / Plot-Window
				\begin{itemize}
				\item Plot \& Tooltip
				\item Markieren \& Löschfunktion
				\item Undo-Funktion
				\item FitLine \& Regressionsgeraden
				\end{itemize}
			\item Filter-Window
				\begin{itemize}
				\item Filtern nach Tag
				\item Filtern im Intervall
				\end{itemize}
			\item Histogramm
				\begin{itemize}
				\item Tages-, Monats- und Jahresgang
				\end{itemize}
			\end{itemize}
    \end{frame}

    \section{Main-Window / Plot-Window}
    \begin{frame}
	    \frametitle{Plot \& Tooltip}
			\begin{itemize}
			\setlength\itemsep{1em}
			\item Darstellung der Datentupel im YX-Punktediagramm
			\item Mausberührung löst einen Tooltip aus
			\end{itemize}
    \end{frame}

    \section{Main-Window / Plot-Window}
    \begin{frame}
			\frametitle{Markieren \& Löschfunktion}
			\begin{itemize}
				\setlength\itemsep{1em}
				\item Punkteauswahl im Plot mit der Maus
				\item Tooltip zeigt Summe, Median, Mittelwert und Varianz
				\item ''Delete''-Knopf löscht ausgewählte Punktemenge
			\end{itemize}
    \end{frame}
    
    \section{Main-Window / Plot-Window}
    \begin{frame}
			\frametitle{Undo-Funktion}
			\begin{itemize}
				\setlength\itemsep{1em}
				\item Änderungen werden rückgängig gemacht
				\item Plot wird neu gezeichnet
				\item Datensatz wird entsprechend angepasst
			\end{itemize}
    \end{frame}


    \section{Main-Window / Plot-Window}
    \begin{frame}
			\frametitle{FitLine \& Regressionsgeraden}
			\begin{itemize}
				\setlength\itemsep{1em}
				\item Berechnung der FitLine mit Hilfe der Least-Square Methode
				\item aktualisiere Regressionsgraph bei Änderung der Auswahl
				\item Füge zwei weitere Geraden mit Parametern ein, zu denen die Fehlerspanne addiert bzw. subtrahiert wird
			\end{itemize}
    \end{frame}


%    \section{Aufgabe 4}
%    \begin{frame}[containsverbatim]
%    	\frametitle{Explorationsloop}
%    	\begin{itemize}
%    		\item
%    	\end{itemize}
%    \end{frame}

    \section{Filter-Window}
    \begin{frame}
			\frametitle{Filtern nach Tag}
			\begin{itemize}
				\setlength\itemsep{1em}
				\item Wähle Monat und Tag im entsprechenden Baum aus
				\item Bei Bestätigung wird der dazugehörige Datensatz im Plot dargestellt
			\end{itemize}
    \end{frame}
    
    \section{Filter-Window}
    \begin{frame}
			\frametitle{Filtern im Intervall }
			\begin{itemize}
				\setlength\itemsep{1em}
				\item Wähle Start- und Enddatum via Slider
				\item Bei Bestätigung wird dazugehörige der Datensatz im Plot dargestellt
			\end{itemize}
    \end{frame}
    
    \section{Histogramm}
    \begin{frame}
			\frametitle{Datenmenge bestimmen}
			\begin{itemize}
				\setlength\itemsep{1em}
				\item Outlier im XY-Punktediagramm bestimmen und markieren				
				\item Histogrammdarstellung in ''Analytics''  wählen (Tages-, Monats- oder Jahresgang)
			\end{itemize}
    \end{frame}
    
    \section{Histogramm}
    \begin{frame}
			\frametitle{Histogramm-Optionen}
			\begin{itemize}
				\setlength\itemsep{1em}
				\item Im Histogramm-Dialog Einstellungen auswählen
				\begin{itemize}
					\setlength\itemsep{1em}
					\item Intervallgröße festlegen 
					\item Monat wählen
					\item Darzustellende Partikel bestimmen				
				\end{itemize}
				\item Tooltip erscheint bei Berührung einer Bar im Histogramm				
			\end{itemize}
    \end{frame}

\end{document}
