\documentclass{beamer}

\usepackage{amsmath}
\usepackage{amssymb}
\usepackage[ngerman]{babel}
\usepackage{graphicx}
\usepackage[T1]{fontenc}
\usepackage[utf8x]{inputenc}

\graphicspath{{./imgs/}}

% customized theme
\usetheme{CambridgeUS}
\usecolortheme{whale}
\beamertemplatenavigationsymbolsempty
\setbeamertemplate{footline}[page number]
\setbeamercolor{frametitle}{fg=white,bg=structure}
\setbeamercolor{theorem}{fg=black,bg=structure!30}
\setbeamercovered{transparent}

% meta informations for title page
\title{Übung 2}
\subtitle{Visual Analytics für raumzeitliche Daten}
\author{Christian Diehr \and Benjamin Drost \and David Foerster}
\institute{Institut für Informatik\\Humboldt-Universität zu Berlin}
\logo{\pgfimage[width=2cm,height=2cm]{./imgs/hulogo}}
\date{03. November 2015}

\begin{document}

    \begin{frame}
        \titlepage
    \end{frame}
    \logo{} % logo only appears on title page
    
    \section{Aufgabe 1}
    \begin{frame}{Datenstruktur(en)}
        \begin{itemize}
	        \setlength\itemsep{1em}
        	\item 
        \end{itemize}
    \end{frame}
    
    \begin{frame}{Funktionalität}
    	\begin{itemize}
    		\setlength\itemsep{1em}
    		\item 
    	\end{itemize}
    \end{frame}

    \section{Aufgabe 2}
    \begin{frame}{Datenstruktur(en)}
    	\begin{itemize}
    		\setlength\itemsep{1em}
    		\item 
    	\end{itemize}
    \end{frame}
    
    \begin{frame}{Funktionalität}
    	\begin{itemize}
    		\setlength\itemsep{1em}
    		\item 
    	\end{itemize}
    \end{frame}

    \section{Aufgabe 3}
    \begin{frame}{Datenstruktur(en)}
    	\begin{itemize}
    		\setlength\itemsep{1em}
    		\item 
    	\end{itemize}
    \end{frame}
    
    \begin{frame}{Funktionalität}
    	\begin{itemize}
    		\setlength\itemsep{1em}
    		\item 
    	\end{itemize}
    \end{frame}

    \section{Aufgabe 4}
    \begin{frame}{Datenstruktur(en)}
    	\begin{itemize}
    		\setlength\itemsep{1em}
    		\item 
    	\end{itemize}
    \end{frame}
    
    \begin{frame}{Funktionalität}
    	\begin{itemize}
    		\setlength\itemsep{1em}
    		\item 
    	\end{itemize}
    \end{frame}
    
    \section{Aufgabe 5}
    \begin{frame}{Datenstruktur(en)}
    	\begin{itemize}
    		\setlength\itemsep{1em}
    		\item 
    	\end{itemize}
    \end{frame}
    
    \begin{frame}{Funktionalität}
    	\begin{itemize}
    		\setlength\itemsep{1em}
    		\item 
    	\end{itemize}
    \end{frame}
    
    \section{Aufgabe 6}
    \begin{frame}{Datenstruktur(en)}
    	\begin{itemize}
    		\setlength\itemsep{1em}
    		\item 
    	\end{itemize}
    \end{frame}
    
    \begin{frame}{Funktionalität}
    	\begin{itemize}
    		\setlength\itemsep{1em}
    		\item 
    	\end{itemize}
    \end{frame}

    \section{Aufgabe 8}
    \begin{frame}{Datenstruktur(en)}
    	\begin{itemize}
    		\setlength\itemsep{1em}
    		\item 
    	\end{itemize}
    \end{frame}
    
    \begin{frame}{Funktionalität}
    	\begin{itemize}
    		\setlength\itemsep{1em}
    		\item 
    	\end{itemize}
    \end{frame}

    \section{Aufgabe 9}
    \begin{frame}{Datenstruktur(en)}
    	\begin{itemize}
    		\setlength\itemsep{1em}
    		\item 
    	\end{itemize}
    \end{frame}
    
    \begin{frame}{Funktionalität}
    	\begin{itemize}
    		\setlength\itemsep{1em}
    		\item 
    	\end{itemize}
    \end{frame}

    \section{Aufgabe 11}
    \begin{frame}{Datenstruktur(en)}
    	\begin{itemize}
    		\setlength\itemsep{1em}
    		\item 
    	\end{itemize}
    \end{frame}
    
    \begin{frame}{Funktionalität}
    	\begin{itemize}
    		\setlength\itemsep{1em}
    		\item 
    	\end{itemize}
    \end{frame}
    
    \section{Leistungen} 
    \begin{frame}{Eigene Leistungen}
    	\begin{itemize}
	    	\setlength\itemsep{1em}
	    	\item Christian
	    	\begin{itemize}
	    		\item Filterfunktion (Aufgabe 6 \& 8)
	    		\item Auswahl von Punkten (Aufgabe 3 \& 4)
	    		\item Historiefunktion (Aufgabe 11)
	    	\end{itemize}
    		\item David
    		\begin{itemize}
    			\item umfangreiches Refactoring
    			\item umfangreiche Verbesserungen (u.a. kD-Baum)
    			\item Skripte für Abgabe
    		\end{itemize}
    		\item Benjamin
    		\begin{itemize}
    			\item Plot (Aufgabe 1 \& 2)
    			\item Timeline (Aufgabe 9)
    			\item Abgabe
    		\end{itemize}
    	\end{itemize}
    \end{frame}
    
\end{document}
