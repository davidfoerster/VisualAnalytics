\documentclass{beamer}

\usepackage{amsmath}
\usepackage{amssymb}
%\usepackage{textcomp}
\usepackage[ngerman]{babel}
\usepackage{graphicx}
\usepackage{color}
\usepackage{listings}
\usepackage{lmodern}
\usepackage[T1]{fontenc}
\usepackage[utf8x]{inputenc}

\graphicspath{{./imgs/}}

% customized theme
\usetheme{CambridgeUS}
\usecolortheme{whale}
\beamertemplatenavigationsymbolsempty
\setbeamertemplate{footline}[page number]
\setbeamercolor{frametitle}{fg=white,bg=structure}
\setbeamercolor{theorem}{fg=black,bg=structure!30}
\setbeamercovered{transparent}

\definecolor{mygreen}{rgb}{0, 0.6, 0}
\definecolor{mymauve}{rgb}{0.58, 0, 0.82}

\lstset{
	language=Python,
	breaklines=true,
	tabsize=4,
	%basicstyle=\ttfamily,
	keywordstyle=\color{blue},
	stringstyle=\color{mymauve},
	commentstyle=\color{mygreen},
	otherkeywords={
		None,False,True,
		self,
		bytes}
}

% meta informations for title page
\title{Übung 2}
\subtitle{Visual Analytics für raumzeitliche Daten}
\author{Christian Diehr \and Benjamin Drost \and David Foerster}
\institute{Institut für Informatik\\Humboldt-Universität zu Berlin}
\logo{\includegraphics[width=2cm,height=2cm]{../hulogo}}
\date{24. November 2015}

\begin{document}

    \begin{frame}
        \titlepage
    \end{frame}
    \logo % logo only appears on title page

    \section{Aufgabe 1}
    \begin{frame}[containsverbatim]
	    \frametitle{Dateneingabe}
      \begin{itemize}
        \setlength\itemsep{1em}
      	\item als Tupel: \lstinline{(bytes(Zeitpunkt), int64(klein), int64(gross))}
      	\item einlesen und parsen mit \lstinline{numpy.genfromtext(...)}
      \end{itemize}
			\begin{lstlisting}
data = numpy.genfromtxt(
	data_pathname,
	dtype = [
		('date', '|S19'),
		('small', 'i8'),
		('large', 'i8')],
	delimiter = ';',
	names = ["date", "small", "large"])
			\end{lstlisting}
    \end{frame}

    \begin{frame}[containsverbatim]
    \frametitle{Plotting}
    	\begin{itemize}
    		\setlength\itemsep{1em}
    		\item darstellen mit \lstinline{pyqtgraph.ScatterPlotItem}
    		\item Achsenbeschriftungen, -abschnitte und Hilfslinien kommen (fast) frei Haus
    	\end{itemize}
    	\begin{lstlisting}
sp = pyqtgraph.ScatterPlotItem(
	data['small'], data['large'],
	pen=None, symbol='o')
form.graphicsView.addItem(sp)
form.graphicsView.setLabel(
	axis='left', text='large')
form.graphicsView.setLabel(
	axis='bottom', text='small')
form.graphicsView.showGrid(True, True)
sp.show()
			\end{lstlisting}
    \end{frame}

    \section{Aufgabe 2}
    \begin{frame}[containsverbatim]
    	\frametitle{Datenstruktur(en)}
    	\begin{itemize}
    		\setlength\itemsep{1em}
    		\item C-Implementierung eines $k$-dimensionalen Suchbaums für Punkte (\lstinline{scipy.spatial.cKDTree})
    	\end{itemize}
    	\begin{lstlisting}
tree = scipy.spatial.cKDTree(
	np.vstack((data['x'], data['y'])).transpose())
nearest_neighbors = tree.query(
	[event.pos()], 1, 0, 2, point_radius)[1]
    	\end{lstlisting}
    	(Achtung: Plot-Kreise sind im Objektraum tatsächlich Ellipsen!)
    \end{frame}

    \begin{frame}[containsverbatim]
    	\frametitle{Tooltip}
    	\begin{itemize}
    		\setlength\itemsep{1em}
    		\item bei Mausbewegung suche Punkt unter Zeiger
    		\item zeichne Infotext neben Punkt
    	\end{itemize}
    	\begin{lstlisting}
	self.tooltip = pyqtgraph.TextItem()

def onMove(self, scene_pos):
	pos = self.scatterpoints.mapFromScene(scene_pos)
	nn = self.scatterpoints.pointsAt(pos)
	if nn is not None:
		s = self.data[nn]
		self.tooltip.setText('...'.format(*s))
		self.tooltip.setPos(s[1], s[2])
		self.tooltip.show()
	else:
		self.tooltip.hide()
    	\end{lstlisting}
    \end{frame}

    \section{Aufgabe 3}
    \begin{frame}[containsverbatim]
    	\frametitle{Einzelauswahl}
    	\begin{itemize}
    		\setlength\itemsep{1em}
    		\item bilde Punktauswahl als Hashset von Tupelindizes ab
    		\item Mausklick fügt Punkt(index) hinzu bzw.\ entfernt einen vorhandenen
    	\end{itemize}
    	\begin{lstlisting}
def mouseClickEvent(self, ev):
	if ev.button() == Qt.LeftButton:
		pt_idx = self.pointsAt(ev.pos())
		if pt_idx is not None:
			pt = self.point(pt_idx)
			selected = self.selection.flip(pt_idx)
			pt.setBrush(
				brush2 if selected else brush1)
			ev.accept()
			return
	ev.ignore()
			\end{lstlisting}
    \end{frame}

    \begin{frame}[containsverbatim]
    	\frametitle{Bereichsauswahl}
    	\begin{itemize}
    		\setlength\itemsep{1em}
    		\item implementiere \lstinline{mouseDragEvent}
    		\item zeichne Auswahlrechteck so lange linke Maustaste gedrückt
    		\item sobald losgelassen, suche darin liegende Punkte im 2-D-Baum (analog zu Einzelauswahl)
    			\begin{itemize}
    				\item \lstinline{cKDTree} kann nur in Kreisen suchen
    				\item suche in Rechteckaußenkreis und filtere Ergebnis
    			\end{itemize}
    		\item füge Punkte der Auswahl hinzu bzw.\ entferne sie bei gedrückter Umschalttaste
    	\end{itemize}
    \end{frame}

    \section{Aufgabe 4}
    \begin{frame}[containsverbatim]
    	\frametitle{Statistik-Tooltip}
    	\begin{itemize}
    		\item Variante des Tooltip-Texts aus Aufgabe 2
    		\item falls Punkt unter Mauszeiger in Auswahl, zeige Statistik zur Menge ausgewählter Punkte
    		\item Berechnung erfolgt bei Bedarf
    			\begin{itemize}
    				\item Ergebnis wird zwischengespeichert
    				\item \ldots und bei Auswahlveränderung verworfen
    				\item realisiert mittels der Erweiterung \lstinline{class DataSelection(set)}
    			\end{itemize}
    	\end{itemize}
    \end{frame}

    \section{Aufgabe 5}
    \begin{frame}{Funktionalität}
    	\begin{itemize}
    		\setlength\itemsep{1em}
    		\item PyQtGraph hat diese Funktionalität bereits:
		  		\begin{description}
						\item[Mauszeiger bei gedrückter rechter Taste ziehen:] stufenlos in die jeweilige Bewegungsrichtung
						\item[Mausrad:] in Stufen bei bestehendem Seitenverhältnis
					\end{description}
    	\end{itemize}
    \end{frame}

    \section{Aufgabe 6}
    \begin{frame}[containsverbatim]
    	\frametitle{Datenstruktur(en)}
    	\begin{itemize}
    		\setlength\itemsep{1em}
    		\item Filter-Fenster erzeugen und TreeWidget füllen
    		\item Filterfunktionalität für Monat und Tag
    	\end{itemize}
    	\begin{lstlisting}
for s in self.new_data:
	decode_date = s[0].decode("utf-8")
	if timeInterval not in decode_date:
		self.new_data = np.delete(self.new_data, index)
	if (len(self.new_data) > 0):
		self.data = self.new_data
		if len(self.data) > 0:
			self.plotFilterRange(...)
		else:
			msgBox.setText("No data exists for the filter!")
    	\end{lstlisting}
    \end{frame}

    \section{Aufgabe 7}
    \begin{frame}{Funktionalität}
    	\begin{itemize}
    		\setlength\itemsep{1em}
    		\item wie Aufgabe 4 mit ein paar zusätzlichen Statistikfuntionen aus dem \lstinline{numpy}-Modul
    	\end{itemize}
    \end{frame}

    \section{Aufgabe 8}
    \begin{frame}[containsverbatim]
    	\frametitle{Datenstruktur(en)}
    	\begin{itemize}
    		\item 2 Filter-Slides im Filter-Fenster hinzufügen
    	\end{itemize}
    	\begin{lstlisting}
for s in self.new_data:
	decode_date = s[0].decode("utf-8")
	if (fromDate in decode_date) and isDeletePoint:
		isDeletePoint = False
	if (toDate in decode_date) and not isDeletePoint:
		isDeletePoint = True
	if isDeletePoint:
		self.new_data = np.delete(self.new_data, index)
	else:
		index += 1
    	\end{lstlisting}
    \end{frame}

    \section{Aufgabe 9}
    \begin{frame}{Datenstruktur(en)}
    	\begin{itemize}
    		\setlength\itemsep{1em}
    		\item Erzeugen der Timeline
    	\end{itemize}
    \end{frame}

    \begin{frame}{Funktionalität}
    	\begin{itemize}
    		\setlength\itemsep{1em}
    		\item
    	\end{itemize}
    \end{frame}

    \section{Aufgabe 11}
    \begin{frame}[containsverbatim]
    	\frametitle{Datenstruktur(en)}
    	\begin{itemize}
    		\setlength\itemsep{1em}
    		\item Undo-Funktion aktuell für max. 3 Schritte
    	\end{itemize}
    	\begin{lstlisting}
def undoFunction(self):
	self.new_data = self.undo_data1
	self.undo_data1 = self.undo_data2
	self.data = self.new_data
	self.plotFilterRange(...)
    	\end{lstlisting}
    \end{frame}

    \section{Leistungen}
    \begin{frame}{Eigene Leistungen}
    	\begin{itemize}
	    	\setlength\itemsep{1em}
	    	\item Christian
	    	\begin{itemize}
	    		\item Filterfunktion (Aufgabe 6 \& 8)
	    		\item Auswahl von Punkten (Aufgabe 3 \& 4)
	    		\item Historiefunktion (Aufgabe 11)
	    	\end{itemize}
    		\item David
    		\begin{itemize}
    			\item Punkt- und bereichsauswahl und deren Darstellung (A.\ 3)
    			\item Refactoring (aufgabenspezifische Klassen statt „Gottobjekt“)
    			\item algorithmische Verbesserungen
    				\begin{itemize}
    					\item k-dimensionaler-Baum
    					\item Hashset statt Liste für ausgewählte Punkte
    					\item zwischengespeicherte, nur bei Bedarf aktualisierte Statistiken über Punktmengen (A.\ 4)
    				\end{itemize}
    			\item Build- und Paktierskripte (für die Abgabe)
    			\item Folieninhalte
    		\end{itemize}
    		\item Benjamin
    		\begin{itemize}
    			\item Plot (Aufgabe 1 \& 2)
    			\item Timeline (Aufgabe 9)
    			\item Abgabe
    		\end{itemize}
    	\end{itemize}
    \end{frame}

\end{document}
